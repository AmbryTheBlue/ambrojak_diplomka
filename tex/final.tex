% !TEX program = optex
% !TEX root = ambrojak-thesis-masters.tex
\label[conclusion]
\chap Conclusion
This thesis addressed the critical need for automated anomaly detection in Metrans transport orders to prevent costly errors and operational disruptions. By analyzing historical order data  and researching existing anomaly detection methods, including statistical approaches, clustering algorithms, and tree-based methods, three distinct approaches were selected and evaluated: Misclassification Detection (MD), Confidence-Based Anomaly Detection (CBAD), and Frequency-Based Outlier Detection (FBOD). The implementation involved preprocessing and vectorizing textual features, generating synthetic anomalies, and building machine learning pipelines using a technical stack including Python, scikit-learn, Keras and MLflow.

The evaluation demonstrated varied performance across six defined anomaly types (heating type, heating temperature, presence of heating request, presence of customs number, presence of additional billing reference, and chassis type). While some anomalies like ``Has Heating'' and ``Additional Billing Reference'' proved relatively trivial with near-perfect accuracy, others, such as ``Temperature'' and ``Chassis Type'', presented more significant challenges, although high F1 scores were still achieved, particularly after handling missing values. The study found that including additional order information (like goods and chassis type for Heating Type anomaly) and utilizing approaches beyond simple MD, such as FBOD and CBAD, generally improved detection performance, with unexpected models like SGD and Logistic Regression performing well in some cases.

The implemented detection models integrate into a modular microservice architecture, deployed via the CodeNOW platform, ensuring scalability, maintainability, and ease of real-time operation. This component is designed to load the best-performing models, process new order data via RabbitMQ events triggered by the Metrans system, and send notifications for human verification when an anomaly is detected. The technical stack for deployment includes Docker, CodeNOW, and REST API (FastAPI). Although the current workflow and model interfaces are functional, further enhancements are crucial for improving both usability and maintainability.
\rfc{Odstavec v závěru o Komponentě kdyžtak mírnit podle toho, jak moc to stihnu.}


\sec Future Work
To further enhance the anomaly detection framework and ensure its long-term applicability and maintainability, the following areas are identified for future development:

A key focus for future work is improving the reusability and adaptability of the developed models and workflows. This involves refactoring the existing codebase to streamline and standardize various components. Specific refactoring tasks include cleaning up development notebooks, organizing functions logically into separate files, and establishing clear, consistent naming conventions for different detection approaches (renaming OODD to FBOD, FallBack to BackOff Smoothing, class_predictors to MD, and probs_predictors to CBAD).

Crucially, the goal is to unify the interface for each anomaly detection approach. This unified interface will be instrumental in improving integration with MLflow, facilitating easier tracking and management of models. Furthermore, a standardized interface will significantly simplify the process of adding new anomaly types to the detection system.

Ultimately, the aim is to wrap the entire anomaly detection workflow into a single, comprehensive function. This function would take the anomaly type as input, automatically evaluate several approaches and models, select the best-performing one, and upload it to MLflow, providing the location of the deployed model as output. This automated workflow will fully prepare the system for periodic manual retraining, allowing the models to be easily updated with new data to maintain their effectiveness over time and adapt to evolving patterns in transport orders.
