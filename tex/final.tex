% !TEX program = optex
% !TEX root = ambrojak-thesis-masters.tex
\label[conclusion]
\chap Conclusion
This thesis addressed the critical need for automated anomaly detection in Metrans transport orders to prevent costly errors and operational disruptions. By analyzing historical order data  and researching existing anomaly detection methods, including statistical approaches, clustering algorithms, and tree-based methods, three distinct approaches were selected and evaluated: Misclassification Detection (MD), Confidence-Based Anomaly Detection (CBAD), and Frequency-Based Outlier Detection (FBOD). The implementation involved preprocessing and vectorizing textual features, generating synthetic anomalies, and building machine learning pipelines using a technical stack including Python, scikit-learn, Keras and MLflow.

The evaluation demonstrated varied performance across six defined anomaly types (heating type, heating temperature, presence of heating request, presence of customs number, presence of additional billing reference, and chassis type). While some anomalies like ``Has Heating'' and ``Additional Billing Reference'' proved relatively trivial with near-perfect accuracy, others, such as ``Temperature'' and ``Chassis Type'', presented more significant challenges, although high F1 scores were still achieved, particularly after handling missing values. The study found that including additional order information (like goods and chassis type for Heating Type anomaly) and utilizing approaches beyond simple MD, such as FBOD and CBAD, generally improved detection performance, with unexpected models like SGD and Logistic Regression performing well in some cases.

The implemented detection models integrate into a modular microservice architecture, deployed via the CodeNOW platform, ensuring scalability, maintainability, and ease of real-time operation. This component is designed to load the best-performing models, process new order data via RabbitMQ events triggered by the Metrans system, and send notifications for human verification when an anomaly is detected. The technical stack for deployment includes Docker and CodeNOW. 

Although parts of the system—particularly the real-time deployment—are still undergoing refinement, the thesis delivers a complete and functional anomaly detection framework ready for further integration. With a modular architecture, reproducible pipelines, and extensible detection strategies, the foundation has been laid for a robust solution that can evolve alongside Metrans’ operational systems. Continued iteration and deployment will ensure that this work provides tangible value in practice, not just in theory.

\rfc{Odstavec v závěru o Komponentě kdyžtak mírnit podle toho, jak moc to stihnu.}


\sec Future Work
To ensure long-term usability and adaptability of the anomaly detection system, several areas have been identified for future development. Most urgently, the real-time detection component requires fixes and refinement. In particular, the mapping between MIS3 and MIS2 schemas must be finished, and the component's interaction with Metrans' queues must be ironed out. Additionally enhanced logging, error handling, and observability will also be welcome improvements for reliable production use.

Once the component is stable, attention can shift to broader refactoring for maintainability. This includes cleaning up development notebooks, restructuring the codebase into well-organized modules, and standardizing naming conventions across detection methods. A unified interface for all detection strategies (MD, CBAD, FBOD) will simplify integration with MLflow and streamline the addition of both new anomaly types and new models. Ultimately, wrapping the workflow into a single callable function, capable of training, evaluating, and recording models based on the type of anomaly, will prepare the system for periodic retraining and future scaling.
