% !TEX program = optex
% !TEX root = ambrojak-thesis-masters.tex

\chap Results
\rfc{Vybrat složitější anomálii, kde se ukáže více typů grafů}
\rfc{Pro ostatní anomálie aspoň nějaké overview}
\rfc{Porovnání heating type a heating type se zbožím, popřípadě jiné anomálie, kde využíváme informace navíc.}
\sec Heating Type
First of lets look at the distribution.

\medskip
\clabel[distribution]{Distribution of anomaly scores}
\picw=15cm \cinspic imgs/distribution.png
\caption/f Distribution of heating types in training data
\medskip


Second lets look at several our FBOD. To test their accuracy I have made artificial anomalies with wrong heating type.
\medskip
\clabel[matrix]{Confusion Matrix}
\picw=15cm \cinspic imgs/Confusion-matrix.png
\caption/f Confusion matrix for several models
\medskip

\medskip
\clabel[F1]{Precision, Recall, F1}
\picw=15cm \cinspic imgs/metrics.png
\caption/f Typical metrics used for anomaly detection
\medskip

\secc Individual model can be further analysed
For models that are using anomaly scoring a more advanced plots can be done. Here are examples on Random Forest classifier.

\medskip
\picw=15cm \cinspic imgs/rf-anomaly-scores.png
\caption/f Distrbution of anomaly scores
\medskip


\medskip
\picw=10cm \cinspic imgs/rf-precision-recall.png
\medskip

\medskip
\picw=10cm \cinspic imgs/rf-roc.png
\medskip

\sec Discussion