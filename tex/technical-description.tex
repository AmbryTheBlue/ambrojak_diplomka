% !TEX program = optex
% !TEX root = ambrojak-thesis-masters.tex
\chap Technical Description
\sec Technical stack used
\rfc{At least add some links to official website/documentation}

\secc Git \& GitLab
In this project, Git and GitLab are used for version control, branching, and delivering code to the cloud to deploy. A short description of them is provided here, but if you are unaware of them, studying them is suggested over reading this piece.

{\bf Git} is a distributed version control system that tracks code changes, supports branching, and enables collaborative development with a decentralized structure, ensuring a full version history and offline work.\urlnote{https://git-scm.com/}

{\bf GitLab} is a web-based platform built on Git that integrates version control, CI / CD, and project management. It supports collaboration, code reviews, and automation, making it ideal for DevOps and team workflow.\urlnote{https://about.gitlab.com/}

\secc Python
Another expected pick is Python as our programming language of choice. Reasons for this are simple. It is widely used. Python has libraries for almost any task that we can expect to want to do.

\secc Docker
Docker was chosen due to avoid the "It works on my machine" issue.

Docker is an open-source platform for developing, deploying, and managing containerized applications. It packages software and its dependencies into lightweight containers, ensuring consistent performance across different environments.

By isolating applications, Docker enhances scalability, portability, and resource efficiency, making it a key tool for modern DevOps practices and cloud-native development.

\secc CodeNow
CodeNOW is a low-code or no-code platform designed to help organizations develop, deploy, and manage cloud-native applications efficiently. It provides tools for automating software development processes, integrating DevOps practices, and managing infrastructure, all while minimizing the need for extensive coding expertise.

CodeNOW is often used to accelerate digital transformation, enabling teams to focus on delivering business value while abstracting the complexity of the underlying systems.

\secc Rest API
A REST API (Representational State Transfer Application Programming Interface) is a web service architecture that allows systems to communicate over HTTP using standard methods like GET, POST, PUT, and DELETE. It is stateless, relying on resource-based URLs and typically exchanging data in formats like JSON or XML.

REST APIs are widely used for their simplicity, scalability, and compatibility, enabling seamless integration between diverse applications and services.

\secc Fastapi
FastAPI is a modern, high-performance web framework for building APIs with Python, leveraging standard Python type hints. It is designed for speed, developer productivity, and ease of use, supporting asynchronous programming and automatic OpenAPI documentation generation.

Built on Starlette and Pydantic, FastAPI enables robust validation, serialization, and high throughput, making it ideal for scalable, production-ready applications.

\secc Keycloak

Keycloak is an open-source identity and access management (IAM) solution that provides authentication, authorization, and user management capabilities. It supports single sign-on (SSO), multi-factor authentication, and integration with identity providers using standards like OAuth2, OpenID Connect, and SAML.

Designed for flexibility, Keycloak simplifies secure application access, enabling centralized user management and seamless integration with modern applications and services.

\secc MLflow

MLflow is an open-source platform for managing the machine learning lifecycle, including experimentation, reproducibility, deployment, and monitoring. It provides tools for tracking experiments, packaging models, and managing model deployments across diverse frameworks and environments.

With its modular design, MLflow supports collaboration and scalability, streamlining workflows for teams working on data-driven projects.

\secc Doris \& Apache Flink
\rfc{Doris a Apache Flink na nic nepoužívám, možná by to chtělo odstranit tuto kapitolu}
{\bf Doris} is an open-source, real-time analytical database that focuses on providing high-performance, low-latency data analytics for large-scale data workloads. Originally developed by Baidu, Doris is designed for OLAP (Online Analytical Processing) use cases, supporting fast querying and seamless data integration. It is known for its simplicity, scalability, and ease of use, with support for real-time data ingestion and querying, making it suitable for analytics platforms, data warehousing, and business intelligence applications.

{\bf Apache Flink} is an open-source stream processing framework for real-time data analytics. It is designed to process large-scale, unbounded data streams with low latency and high throughput. Flink supports both batch and stream processing and provides advanced features like stateful computations, event time processing, and windowing. It is highly scalable, fault-tolerant, and integrates well with other data systems like Apache Kafka, Hadoop, and various data stores. Flink is commonly used in applications requiring real-time analytics, monitoring, and event-driven processing.