% !TEX program = optex
% !TEX root = ambrojak-thesis-masters.tex
\chap Methods for Anomly Detection
\rfc{This chapter (Theorethical background and approaches) should probably be revised. Maybe moved after anomaly and stated which models to use based on our data or keep it here and more thereical/general? But I do not like the writing style very much.}
\sec Anomaly Detection
Anomaly detection, also referred to as outlier detection, is the task of identifying instances in a dataset that deviate significantly from the majority of the data. These deviations, termed anomalies, often represent rare or unusual events, errors, or significant phenomena of interest. Applications of anomaly detection span diverse fields, including fraud detection, system monitoring, medical diagnostics, and natural language processing.


\secc Unsupervised Anomaly Detection
Unsupervised anomaly detection is particularly relevant in cases where labeled data (i.e., explicit examples of anomalous and normal instances) is unavailable. In this paradigm, the algorithm assumes that anomalies are rare and distinct from the "normal" data. The approach involves modeling the patterns, distributions, or structure of normal data and identifying data points that deviate significantly from these learned patterns.

This method is especially suited for scenarios where only examples of normal behavior are available. By leveraging techniques such as clustering, reconstruction errors, distance measures, or density estimation, unsupervised anomaly detection provides a robust framework for identifying unexpected patterns without relying on prior knowledge of what constitutes an anomaly.

\sec Statistical Approaches to Anomaly Detection

Statistical methods for anomaly detection rely on the assumption that the data follows a specific distribution or statistical model. Anomalies are identified as points that deviate significantly from this assumed distribution or exhibit rare or improbable characteristics. These methods are particularly effective when the underlying data structure is well understood and the anomalies are clearly distinguishable from normal variation.

\secc Parametric Approaches
These methods assume that the data conforms to a known probability distribution, such as Gaussian or Poisson. Key techniques include:

\begitems
* {\bf Z-Score Analysis:}\nl
Measures how far a data point is from the mean in terms of standard deviations. Points with high absolute z-scores (e.g., >3) are flagged as anomalies.


Effective for continuous, normally distributed data.

* {\bf Probability Density Estimation (PDE):}\nl
Anomalies are identified as points with very low likelihood under the fitted distribution (e.g., using maximum likelihood estimation).
* {\bf Chi-Square Test:}\nl
Compares observed and expected frequencies for categorical data. Significant deviations indicate anomalies.
\enditems

\secc Non-Parametric Approaches
These methods do not make strong assumptions about the data distribution:

\begitems
* {\bf Kernel Density Estimation (KDE):}\nl
Estimates the probability density function of the data. Points in regions of low density are flagged as anomalies.

* {\bf Nearest Neighbor-Based Methods:}\nl
Compute distances between data points. Anomalies are far from their nearest neighbors in the feature space.

Example: {\em Local Outlier Factor} (LOF), which uses local density to detect outliers.
\enditems

\secc Multivariate Statistical Techniques
For data with multiple variables, relationships among dimensions are considered:

\begitems
* {\bf Mahalanobis Distance:}\nl
Measures the distance of a point from the mean while accounting for the covariance structure of the data. Points with high distances are anomalies.


Effective for identifying anomalies in multivariate normal distributions.


* {\bf Principal Component Analysis (PCA):}\nl
Reduces the dimensionality of the data, capturing the most significant variance. Points with high reconstruction errors (i.e., poorly explained by principal components) are flagged as anomalies.
\enditems

\secc Time-Series-Specific Methods
For temporal data, statistical methods often account for sequential dependencies:

\begitems
* {\bf Moving Average and Exponential Smoothing:}\nl
Detect sudden deviations from expected values derived from past trends.

* {\bf ARIMA Models:}\nl
Use autoregressive integrated moving average models to forecast expected values and flag deviations.
\enditems

\secc Strengths and Limitations
{\em Strengths:}
\begitems
 * {\bf Interpretability:} Statistical methods provide clear thresholds or probabilistic interpretations.
* {\bf Simplicity:} Often computationally efficient and easy to implement.
\enditems
{\em Limitations:}
\begitems
* {\bf Sensitivity to Assumptions:} Performance depends on the validity of the assumed data distribution.
* {\bf Scalability:} May struggle with large or high-dimensional datasets without preprocessing.
* {\bf Handling Complex Relationships:} Struggle with non-linear or intricate dependencies without adaptations.
\enditems

Statistical methods are foundational in anomaly detection and are often used as benchmarks or in conjunction with more advanced machine learning techniques.



\sec Clustering-Based Approaches to Anomaly Detection

Clustering-based anomaly detection methods group data points into clusters based on their similarity and identify anomalies as points that do not belong to any cluster or are significantly far from cluster centroids. These methods are unsupervised, making them suitable for scenarios where no labeled data is available.

\secc Core Principles
Cluster Membership:
\begitems
* Normal data points form dense, well-defined clusters.
* Anomalies are isolated points or lie in low-density regions.
\enditems
Distance to Clusters:
\begitems
* Anomalies are distant from cluster centroids or boundaries.
\enditems


\secc K-Means
Methodology:
\begitems
* Assigns data points to k clusters by minimizing the distance to the nearest cluster centroid.
* Anomalies are points with high distances from their assigned cluster centroid.
\enditems
\nl
Strengths:
\begitems
* Simple and computationally efficient.
\enditems
\nl
Limitations:
\begitems
* Requires predefining the number of clusters (k).
* Assumes clusters are spherical and equally sized, which may not align with real-world data.
\enditems
\secc DBSCAN (Density-Based Spatial Clustering of Applications with Noise)
Methodology:
\begitems
* Groups points into clusters based on local density. Points in sparse regions are classified as noise (anomalies).
\enditems
\nl
Strengths:
\begitems
* Handles arbitrary cluster shapes.
* Does not require specifying the number of clusters.
\enditems
\nl
Limitations:
\begitems
* Sensitive to the choice of parameters (e.g., neighborhood radius ε and minimum points).
* Struggles with varying density levels.
\enditems

\secc Hierarchical Clustering
Methodology:
\begitems
* Constructs a hierarchy of clusters via agglomerative (bottom-up) or divisive (top-down) methods.
* Anomalies are points that remain unclustered at a specific level of the hierarchy or form singleton clusters.
\enditems
\nl
Strengths:
\begitems
* Provides a flexible structure for identifying outliers at various levels.
\enditems
\nl
Limitations:
\begitems
* Computationally expensive for large datasets.
\enditems

\secc Gaussian Mixture Models (GMM)
Methodology:
\begitems
 * Assumes data is generated from a mixture of Gaussian distributions.
* Anomalies are points with low probabilities under the fitted Gaussian components.
\enditems
\nl
Strengths:
\begitems
* Probabilistic foundation provides interpretable scores.
\enditems
\nl
Limitations:
\begitems
* Assumes Gaussian clusters, which may not suit all data.
\enditems

\secc Spectral Clustering
Methodology:
\begitems
* Constructs a similarity graph and uses its eigenvalues to form clusters.
* Points with weak connections to any cluster are flagged as anomalies.
\enditems
Strengths:
\begitems
* Effective for non-linear cluster structures.
\enditems
Limitations:
\begitems
* Computationally intensive for large datasets.
\enditems

\secc Strengths and Limitations
{\em Strengths:}
\begitems
* {\bf Unsupervised:} No need for labeled data.
* {\bf Versatile:} Handles various types of data.
* {\bf Intuitive:} Anomalies are naturally defined by distance or density.
\enditems
{\em Limitations:}
\begitems
* {\bf Parameter Sensitivity:} Performance depends on parameters like k, ε, or minimum density.
* {\bf High Dimensionality:} Clustering algorithms often struggle in high-dimensional spaces ("curse of dimensionality").
* {\bf Scalability:} Methods like hierarchical or spectral clustering are computationally expensive for large datasets.
\enditems

Clustering-based methods are widely used in anomaly detection and are often combined with feature engineering or dimensionality reduction to improve performance.

\sec Isolation Forest and Tree-Based Methods for Anomaly Detection

Tree-based methods leverage the structure of decision trees to isolate anomalies based on their distinct characteristics. These techniques are particularly suited for unsupervised anomaly detection, as they do not rely on labeled data and work well with mixed data types.

\secc Isolation Forest
Isolation Forest ({\em iForest}) is explicitly designed for anomaly detection. It operates on the principle that anomalies are easier to isolate than normal points because they are rare and distinct.
\nl
{\em Description of the core algorithm}
\begitems
* {\bf Random Partitioning:} Randomly select a feature and split the dataset at a randomly chosen value within the feature's range.
* {\bf Recursive Splitting:} Build a tree by recursively partitioning the data until all points are isolated or a maximum tree depth is reached.
* {\bf Anomaly Scoring:} The depth of a point in the tree (number of splits required to isolate it) is inversely proportional to its anomaly score. Points isolated in fewer splits (shorter depth) are considered anomalous.
\enditems
\nl
Strengths:
\begitems
* {\bf Computational Efficiency:} Works in linear time with respect to the number of data points.
* {\bf Handles Mixed Data:} Suitable for both categorical and numerical features.
* {\bf Scalable:} Works well with large datasets.
\enditems
\nl
Limitations
\begitems
* Requires parameter tuning (e.g., number of trees, subsampling size).
* May struggle with data where anomalies closely resemble the majority class.
\enditems

\secc Decision Trees for Anomaly Detection
{\bf Supervised Context: }
When labeled data is available, decision trees (e.g., CART, C4.5) can classify anomalies explicitly. Fit a tree to distinguish between "normal" and "anomalous" data. Less interpretable in the presence of subtle anomalies.


{\bf Unsupervised Context: }
Adapt decision tree logic to identify patterns in the majority class and flag outliers. Use thresholds for decision paths where certain leaf nodes represent unusual patterns. Random splits or density-based measures are employed to enhance detection.
\nl
Strengths
\begitems
* Easy to interpret, especially for small trees.
* Handles both numerical and categorical data.
\enditems
Limitations
\begitems
* Can overfit without proper pruning or parameter tuning.
* Less robust to high-dimensional data compared to ensemble methods like Isolation Forest.
\enditems

\secc Random Forests for Anomaly Detection
Random forests, while typically used for classification or regression, can be adapted for anomaly detection:
\begitems
* Compute leaf nodes for each tree and measure how often a data point lands in rare or low-density leaves.
* Combine these measures to produce an anomaly score.
\enditems
\nl
Strengths:
\begitems
* Benefits from ensemble robustness.
* Can handle diverse types of data.
\enditems
\nl
Limitations
\begitems
* Computationally more expensive than simpler tree-based models.
* Requires careful interpretation of output anomaly scores
\enditems

\secc Overview 
Tree-based methods are particularly valuable when interpretability and computational efficiency are critical. Isolation Forest, in particular, is a strong baseline for anomaly detection in many scenarios.
\nl
\cite[aima]\cite[anomaly]