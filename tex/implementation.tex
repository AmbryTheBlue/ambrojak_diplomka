% !TEX program = optex
% !TEX root = ambrojak-thesis-masters.tex
\label[implementation]
\chap Implementation and System Architecture
\sec Technical stack used

\secc Python, Jupyter, scikit-learn, Keras
Unexpectadly Python was selected as our programming language of choice. The reasons for this are quick development, flexibility, and that it has many ready-to-use machine learning libraries. For example scikit-learn is an ``open source, commercially usable machine learning library built on NumPy, SciPy, and matplotlib.''\urlnote{https://scikit-learn.org/stable/index.html}

Another library we have used is Keras that is very useful for Neural Networks.\urlnote{https://keras.io/} And for quick testing Jupyter Notebooks have been used. They were especailly usefull in combination with JupyerHub that allowed me to work on the servers with access to data and I did not have to download them manually.\urlnote{https://jupyter.org/hub}

\secc MLflow
MLflow is an open-source platform for managing the machine learning lifecycle, including experimentation, reproducibility, deployment, and monitoring. It provides tools for tracking experiments, packaging models, and managing model deployments across diverse frameworks and environments.

With its modular design, MLflow supports collaboration and scalability, streamlining workflows for teams working on data-driven projects.\urlnote{https://mlflow.org/docs/latest/index.html}

\secc Git \& GitLab
In this project, Git and GitLab are used for version control, branching, and  saving code to ``cloud'' and synchronization. A short description of them is provided here. \fnote{but if you are unaware of them, studying them is suggested over reading this piece.}

{\bf Git} is a distributed version control system that tracks code changes, supports branching, and enables collaborative development with a decentralized structure, ensuring a full version history and offline work.\urlnote{https://git-scm.com/}

{\bf GitLab} is a web-based platform built on Git that integrates version control, CI / CD, and project management. It supports collaboration, code reviews, and automation, making it ideal for DevOps and team workflow.\urlnote{https://about.gitlab.com/}

\sec My models and pipelines
\rfc{V téhle sekci bych chtěl přidat ukázky kódu, pipeline popřípadě nějaký diagram? Dává Smysl?}

\secc Preprocessing

\secc Vecorization

\secc Synthetic anomaly generators

\secc Pipeline

\secc Steps Combined