% !TEX program = optex
% !TEX root = ambrojak-defense-masters.tex
\input ctustyle3

\let\s=\relax \let\restore=\relax

\worktype[D/EN]
\faculty{F3}
\department{Department of Computer Science}

\activettchar`

\slides

\slideshow 

\tit Anomaly Detection in Metrans Transport Orders

\subtit Masters thesis defense

\bigskip
\subtit\Black Jakub Ambroz\nl ambrojak@fel.cvut.cz

% \subtit\rm \url{http://petr.olsak.net/ctustyle.html}

\pg; %------------------------------------------------------------------

\sec METRANS and Multimodal Transport

* Transport of ISO containers throughout central and eastern europe

* Mainly railway (140 locomotives, 4000 containers) and trucks in the “Last mile”

\pg;
\sec Anomaly types

* Heating Type (Hot Water, Steam, Electric)

* Heating Temperature (Medium Temperature, Product Temperature From/To)

* Presence of Heating Request

* Presence of Customs Number

* Presence of Additional Billing Reference

* Chassis Type

\pg;
\sec Data
* Text preprocessing of both input and “output” columns
\begitems
* Sender and recipient names, goods description, destination address
\enditems

* Bag of Words for vectorization of textual fields

* Label Encoder for “output” variable

* Abundance of “ok” data, almost no anomalous data

\pg;
\sec Misclassification Detection (MD)
* Idea: train classifier to predict “output” variable

* On mismatch between reality and prediction flag datum as anomalous

$$\hbox{ Anomaly detected if:  } C(x) \neq y \eqmark[eq-MD]$$

* Issues when multiple classes can be correct. Especially with 2+ classes.

\pg;
\sec Confidence Based Anomaly Detection
* Idea: Use “confidence” of the classifier for observed class

$$\hbox{Anomaly detected if } P(y | x) < \tau \eqmark[eq-CBAD]$$

where $P(y∣x)$ is the probability assigned to the true class $y$ for observation $x$ by a trained classifier $C$, and $\tau$  is a predefined threshold.

\pg;
\sec Frequency Based Outlier Detection
* Idea: Use probability of occurrence in training data of observed class $y$ given features $F$, flag as anomalous if it is below threshold $\tau$

$$
\hbox{Anomaly detected if } {{\rm Count}(F, y)\over {\rm Count}(F) + 1} < \tau \eqmark[eq-FBOD]$$ 


\secc Back-off Smoothing
* Idea: When using many features the combination will get increasingly rare. Use smaller subsets.

From features $F=(f_1, f_2, f_3)$ try the following combinations until hit:
$$
(f_1, f_2, f_3), (f_1, f_2), (f_1, f_3), (f_2, f_3), (f_1), (f_2), (f_3)
$$

\pg;
\sec Evaluation
* Models compared using F1 score
$$\rm
{Precision} = {TP \over TP + FP} \eqmark[eq-precision]
$$
$$\rm
{Recall} = {TP \over TP + FN} \eqmark[eq-recall]
$$
$$\rm
{F1} = 2 \times { Precision \times Recall \over Precision + Recall} \eqmark[eq-f1]
$$
* Confidence based models can be further analysed using Anomaly Score Distribution, ROC Curve and Precision Recall curve

\medskip
\centerline{\picw=6cm \inspic imgs/AnomalyScores.pdf \inspic imgs/ROCcurve.pdf \inspic imgs/Precision-Recall.pdf  } 

\pg;
\sec Results
* Excerpt from a table comparing models
\midinsert \clabel[tableChassisimport]{Model Comparison Chassis import}
\ctable{l|rr|r}{
Model & Rec & Prec & F1 \crli \tskip4pt
CBAD(t=0.1):sgdLogLoss & 0.985 & 0.974 & 0.980 \cr
FBODBackOff-nopreprocess & 0.981 & 0.946 & 0.963 \cr
CBAD(t=0.1):KNeighborsClassifier & 0.969 & 0.935 & 0.952 \cr
... & ...  & ... & ... \cr
MD:KerasNNClassifier & 0.996 & 0.728 & 0.841 \cr
MD:rf & 0.996 & 0.728 & 0.841 \cr
FullFBODBackOff & 0.749 & 0.933 & 0.831 \cr
... & ...  & ... & ... \cr
CBAD(t=0.9):ComplementNB & 0.999 & 0.525 & 0.688 \cr
CBAD(t=0.9):sgdLogLoss & 0.999 & 0.512 & 0.677 \cr
}
\caption/t Performance metrics per model in Chassis import (sorted by F1)
\endinsert

\pg;
\sec Conclusion
* Pipeline for training several types of models with different approaches was created
* Models are compared and best is uploaded for use on new data
* F1 score upwards 0.99 usually, 0.98 reached always
* Component for live Anomaly Detection in Metrans IT systems still has a few issues (MIS3-to-MIS2 mapping, message queue integration)

\pg;

\null
\vskip2cm
\centerline{\typosize[35/40]\bf Thank you for your attention}\pg+

\vskip2cm
\centerline{\Blue\typosize[60/70]\bf Questions?}

\pg;

\sec Supervisor's comments
* Mistakes in labels in graph labels: one-off-numbering (1,2,3 -> 0,1,2), overlapping (unreadable) distribution labels
\begitems
* My personal style should not have passed correction and made it to final version.
\enditems


* Awkward, unprofessional phrasing (e.g. “Unexpectedly Python was selected as our programming language of choice.”)
\begitems
* My personal style should not have passed correction and made it to final version.
\enditems

* Reason for Docker
\begitems
* Saying that it is restriction imposed by METRANS' architecture would have been sufficient. Instead I spent too much time explaining why it is used etc. Happened for other technologies.
\enditems

\pg;
\sec Opponent's comments 1

* “The student employed expertise in the field of his study. He also clearly described how he approached the problem and how the problem was solved. However, the problem could have been stated more clearly with some examples. The introduction could have included some naïve method with rule-based engine filtering the data containing anomalies, how this manual process and its configuration slows down the process. It could also serve as a baseline to compare results with.”

\begitems
* I did not try to manually code special if statements. While doable for smaller anomalies like "Heating Type" it would still lack in robustness. And it is unviable approach for “bigger” anomalies - too many senders, recipients.
\enditems

\pg;
\sec Opponent's comments 2
* “Given the volume of the data and given the fact that orders are created by humans with their tendency to make errors, does it make sense to notify the users about the anomalies via email?”
\begitems
* cca 100-200 thousand orders per year: cca 300-500 orders per day
* Historical error rate cca 1-2 \%
* Not all are the types of anomalies we were trying to catch!
* At most around 10 emails per day for manual recheck
* {\bf Given the potential losses it is worth the manual check.}
* It does not have to be email necessarily, the notification for human double check could be made via different channel
\enditems
\pg;

\pg;

\sec Other Results
* Excerpt from a table comparing models

\midinsert \clabel[tableTEMPERATUREwithinfo]{Model Comparison TEMPERATURE with info}
\ctable{l|rr|r}{
Model & Rec & Prec & F1 \crli \tskip4pt
CBAD(t=0.1):LogRegression & 1.000 & 0.989 & 0.994 \cr
CBAD(t=0.1):sgdModifiedHuber & 1.000 & 0.988 & 0.994 \cr
FBODBackOff & 0.998 & 0.988 & 0.993 \cr
CBAD(t=0.1):sgdLogLoss & 0.994 & 0.988 & 0.991 \cr
FBOD & 0.994 & 0.985 & 0.990 \cr
... & ...  & ... & ... \cr
MD:rf & 1.000 & 0.950 & 0.974 \cr
MD:KerasNNClassifier & 1.000 & 0.948 & 0.973 \cr
MD:KNeighborsClassifier & 0.999 & 0.948 & 0.973 \cr
CBAD(t=0.1):VotingClassifier & 0.957 & 0.989 & 0.973 \cr
... & ...  & ... & ... \cr
CBAD(t=0.9):LogRegression &  1.000 & 0.794 & 0.885 \cr
CBAD(t=0.9):sgdLogLoss & 1.000 & 0.776 & 0.874 \cr
}
\caption/t Performance metrics per model in TEMPERATURE with info (sorted by F1)
\endinsert

\sec TODO DELETE ALL BELOW

* The document is included in a file (say `file.tex`)\nl
  and it can be processed by `optex file` command.
* The header of the document should be:

\begtt
\input ctustyle3  % CTUstyle macro for OpTeX
\slides           % slides initialization
\worktype[B/EN]   % type of the work (B,M,D,O) and language (CZ,SK,EN)
\faculty{F3}      % the faculty in short
\department {Department of Mathematics} % department

\slideshow        % begin of the document
... document ...
\pg.
\endtt

* The document must be finished by `\pg` followed by period.

* You need OpTeX in the version Jun~2020 or newer.\nl
  See \url{http://petr.olsak.net/optex}.

* The work type should be set similarly as in {\bf\Blue CTUstyle}.

* Only `\worktype`, `\faculty` and `\department`
  work here. No more declaration sequences from {\bf\Blue CTUstyle}.

\pg; %------------------------------------------------------------------

\sec The structural commands

* You can type \code{\*} for starting of the item.
* Nested items lists (second and more level) are created in
  the `\begitems`\dots`\enditems` environments.
* The slide titles are created by `\sec Text` 
  followed by end of line.\nl 
  For subsections, you can use `\secc Text` similarly.
* The title page (first slide) can be special if `\tit Title`\nl
  (followed by end of line) is used here.
* The `\subtit Author name etc.` (followed by end of line)
  can be used after `\tit` at the first slide.
* The paragraph texts are ragged right.
* You can use `\nl` for new line in the paragraph or titles.
* You can use `\pg` followed by \code{+} or \code{;} or \code{.}
  for new slide.
* The page-bar in the right corner is clickable and it will be 
  created correctly after second pass of the \TeX{} run.

\pg; %------------------------------------------------------------------

\sec Next page (next slide)

* The control sequence \code{\\pg} must be followed by:

\begitems
* the character \frame{\strut\code{+}} -- next page keeps the same text
  and a next text is added (usable for partially uncovering of ideas),\pg+ 
* the character \frame{\strut\code{;}} -- normal next page,\pg+
* the character \frame{\strut\code{.}} -- the end of the document.
\enditems
\pg+

* Summary:
\begtt
\pg+    ... uncover next text
\pg;    ... next page
\pg.    ... the end of the document
\endtt
\pg+

* If the control sequence `\slideshow` is removed (or commented out)
  from the beginning of the document then `\pg+` sequences are deactivated.
  This is usable for printing version of the document.

\pg; %------------------------------------------------------------------

\sec Verbatim

\secc Verbatim in paragraph

* Unlike {\Blue\bf CTUstyle} for OPmac, you can use 
  `"code text"` inside paragraph directly.
* If you declare \code{\\activettchar`} before `\slides` then you can use
  \code{`code text`} like in Markdown. 
* You can use `\code{text}` too.
* All these features are described in \OpTeX/ documentation. 

\pg+

\secc Multi-line verbatim

* Unlike {\Blue\bf CTUstyle} for OPmac, you can use the pair
  `\begtt...\endtt` directly as described in \OpTeX/ documentation.
  No `\pg=` is needed.
* Of course, you can use `\verbinut` too, if you want.

\pg; %------------------------------------------------------------------

\sec Example of multi-line verbatim

The source code includes:

\begtt \adef|{\_bslash}
|begtt |hisyntac{C}
#include <stdio.h>
int main(); // This is a program in C
{
  printf("Hello world!\n");
}
|endtt
\endtt

\pg+

and the result is:

\begtt \hisyntax{C}
#include <stdio.h>
int main(); // This is a program in C
{
  printf("Hello world!\n");
}
\endtt

Note that local declarations can be inserted after `\begtt`,\nl
the `\hisyntax` declaration is used in the example here.

\pg; %------------------------------------------------------------------

\sec Limits of the \code{\\pg+} sequence

* The \code{\\pg+} sequence cannot be used inside a group.
* The exception is the nested environment \code{\\begitems...\\enditems}.\pg+

\secc What to do?

* If you need to partially uncover the multi-line verbatim\nl
  then you can use:

\begtt \adef/{\bslash}
\begtt
... first line of the code ...
/endtt 
\par\pg+ \vskip-6.75pt
\begtt
... second line of the code ...
/endtt
\endtt

\pg+


* If you need to uncover the texts more ingenious then you can
  use `\layers...\endlayers` environment (see next slide\dots)

\pg; %------------------------------------------------------------------

\sec Uncovering by `\layers`, `\pshow`

* You can declare layers inside a slide by `\layers` $n$ ... `\endlayers` pair.
  The number $n$ declares the number of layers.

* The page with `\layers...\endlayers` pair is repeated $n$-times.

* You can use `\pshow` $k$ inside `\layers...\endlayers` environment.\nl
  This macro means {\em partially show}. It prints the following text to the end
  of current group:

\begitems
* invisible, if the number of the slide layer is less than $k$,
* red, if the number of slide the layer is equal to $k$,
* normal (black), if the number of slide layers is greater than $k$. 
\enditems

* The verbatim text and `\secc` macros canot be used inside 
  `\layer` environment.

* See \OpTeX/ documentation for more information.

* Next slide shows the usage of `\pshow`.

\pg; %------------------------------------------------------------------

\sec Example of `\pshow` usage

\setbox0=\vbox{
\begtt
\secc Ideas in special order

\layers 3
* {\pshow1 First idea}
* {\pshow3 Second idea}
* {\pshow2 Third idea}
\endlayers
\pg+

\secc A formula

\layers 4
Consider
$$ 
  E = {\pshow2 m}{\pshow3 c^2}
$$
\endlayers
That is great!
\pg;

\endtt
}
\puttext 100mm -120mm {\box0}

\bgroup
\hsize=8cm

\secc Ideas in special order

\layers 3
* {\pshow1 First idea}
* {\pshow3 Second idea}
* {\pshow2 Third idea}
\endlayers
\egroup
\pg+

\bgroup
\hsize=8cm 
\secc A formula
\egroup

\layers 4
\bgroup
Consider \hsize=8cm
$$ 
  E = {\pshow2 m}{\pshow3 c^2}
$$
\egroup
\endlayers
That is great!


\pg; %------------------------------------------------------------------

\restore

\sec Tables, pictures

* Tables can be created by `\table` or `\ctable` macro.
* Pictures can be included by \code{\\inspic} macro.
* See {\Blue\bf CTUstyle} and \OpTeX/ documentation for more details.
* The centering would be done by the `\centerline` macro.
* Example:\pg+

\begtt
\centerline{\picw=5cm \inspic imgs/AnomalyScores.pdf }

\endtt

\medskip
\centerline{\picw=5cm \inspic imgs/ROCcurve.pdf } 

\pg+

* You can use `\puttext` or `\putpic` macro for arbitrary positioning of
  texts or images. 

\pg; %------------------------------------------------------------------

\sec Comparison CTUslides with Beamer\fnote{\url{http://www.ctan.org/pkg/beamer}}

The \LaTeX{} package Beamer gives much more features and many themes
are prepared for Beamer, {\bf\Red but}

\pg+
* the user of Beamer is forced to {\em program} his/her document using 
  dozens of \code{\\begin{foo}} and \code{\\end{foo}} and many another
  programming constructions,
* on the other hand, plain \TeX{} gives you a possibility to simply 
  {\em write} your document with minimal markup. The result is more compact.
\pg+
* You need to read 250 pages of doc for understanding Beamer,
* on the other hand, you need to read only ten 
  slides\fnote{this eleventh slide isn't counted}
  and you are ready to use {\bf\Blue CTUslides}.
\pg+
* A notice for programmers: to create another individual typographical 
  design for \LaTeX{} is much more complicated than to do the same
  in plain \TeX. And you need to seriously understand plain \TeX{} if you
  want to do something more complicated in \LaTeX.

\pg; %------------------------------------------------------------------

\null
\vskip2cm
\centerline{\typosize[35/40]\bf Thanks for your attention}\pg+

\vskip2cm
\centerline{\Blue\typosize[60/70]\bf Questions?}

\pg. %------------------------------KONEC-------------------------------

