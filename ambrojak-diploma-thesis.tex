% !TEX program = optex
% !TEX root = ambrojak-diploma-thesis.tex
% The documentation of the usage of CTUstyle -- the template for
% typessetting thesis by plain\TeX at CTU in Prague
% ---------------------------------------------------------------------
% Petr Olsak  Jan. 2013

% You can copy this file to your own file and do some changes.
% Then you can run:  optex your-file

\input ctustyle3  % The template (in version 3, for OpTeX) is included here.

\worktype [O/EN] % Type: B = bachelor, M = master, D = Ph.D., O = other
                 % / the language: CZ = Czech, SK = Slovak, EN = English

\faculty    {F3}  % Type your faculty F1, F2, F3, etc. or MUVS
            % use main language of your document here:
\department {Department of Computer Science}
\title      {Processing and data mining from documents for Metrans company using NLP}
\subtitle   {Software or Research project}
\subtitleCZ   {Softwarový nebo výzkumný projekt}
            % \subtitle is optional
\author     {Jakub Ambroz}
\date       {September 2024, January 2025}
\supervisor {Ing. Martin Komárek}  % One or more supervisors
\studyinfo  {Open informatics, specialization Data science}  % Study programme etc.
\workname   {Software or Research project} % Used only if \worktype [O/*] (Other)
            % optional more information about the document:
\workinfo   {}
            % Title / Subtitle in minor language:
\titleCZ    {Vytěžování a zpracování dokumentů pro firmu Metrans pomocí NLP}
\subtitleEN {the \OpTeX/ template for theses at CTU}
            % If minor language is other than English
            % use \titleCZ, \subtitleCZ or \titleSK, \subtitleSK instead it.
\pagetwo    {}  % The text printed on the page 2 at the bottom.

\abstractEN {
   
   This document is a report for course B4MSVP (Software or research project) but it is written as a part of the following Diploma thesis.
}
\abstractCZ {

}
   

\keywordsEN {%
NLP, Anomaly detection, 
}
\keywordsCZ {%
NLP,
}
\thanks {           % Use main language here
   I would like to thank Stratox. Especially Ing. Martin Komínek for his support with all technical issues I encountered. And Metrans for providing the necessary real world data.
}
\declaration {      % Use main language here
% TODO: CHANGE TO ENLGISH
   Prohlašuji, že jsem předloženou práci vypracoval
   samostatně a že jsem uvedl veškeré použité informační zdroje v~souladu
   s~Metodickým pokynem o~dodržování etických principů při přípravě
   vysokoškolských závěrečných prací.

   V Praze dne 19.01.2025 % !!! Attention, you have to change this item.
   \signature % makes dots
}

%%%%% <--   % The place for your own macros is here.

%\draft     % Uncomment this if the version of your document is working only.
%\linespacing=1.7  % uncomment this if you need more spaces between lines
                   % Warning: this works only when \draft is activated!
%\savetoner        % Turns off the lightBlue backround of tables and
                   % verbatims, only for \draft version.
%\blackwhite       % Use this if you need really Black+White thesis.
%\onesideprinting  % Use this if you really don't use duplex printing. 
%\totf             % More compact list of tables and figures.

\makefront  % Mandatory command. Makes title page, acknowledgment, contents etc.

\input introduction

% \input uvod    % Files where the source of the document is prepared.
% \input popis   % Full name is: uvod.tex, popis.tex, the suffix can be omitted.
% \input prilohy

\bibchap

\bye
