\glos {BHT} {Bremer Hafentelematik — an electronic system used for managing export declarations and port clearance in the ports of Bremen and Bremerhaven}
\glos {ZAPP} {Zoll-Ausfuhrüberwachung im Paperless Port — a customs export monitoring platform used in the Port of Hamburg for digital handling of export processes}
\glos {TCC} {Terminal Customs Code — a unique code required for customs clearance at certain ports (e.g., Koper, Trieste), used to authorize container handling}
\glos {HS Code} {Harmonized System code — an internationally standardized system of names and numbers used to classify traded products for customs and statistical purposes}
\glos {HS} {See HS Code}
\glos {CMR} {``Convention relative au contrat de transport international de marchandises par route'' — the official French title of the 1956 United Nations treaty governing international road freight transport. In English, it is known as the Convention on the Contract for the International Carriage of Goods by Road. he CMR waybill, issued under this convention, serves as a legally binding transport document outlining the rights and obligations of the sender, carrier, and recipient. It is widely used across Europe and other signatory countries to ensure uniformity and legal clarity in international road logistics}
\glos {TEU}{TEU stands for twenty-foot equivalent unit, a general unit of cargo capacity for container ships and ports.}
