% !TEX program = optex
% !TEX root = ambrojak-thesis-masters.tex
% The documentation of the usage of CTUstyle -- the template for
% typessetting thesis by plain\TeX at CTU in Prague
% ---------------------------------------------------------------------
% Petr Olsak  Jan. 2013

% You can copy this file to your own file and do some changes.
% Then you can run:  optex your-file

\input ctustyle3  % The template (in version 3, for OpTeX) is included here.

\worktype [M/EN] % Type: B = bachelor, M = master, D = Ph.D., O = other
                 % / the language: CZ = Czech, SK = Slovak, EN = English

\faculty    {F3}  % Type your faculty F1, F2, F3, etc. or MUVS
            % use main language of your document here:
\department {Department of Computer Science}
\title      {Anomaly Detection in Metrans Transport Orders}
\subtitle   {Misclassification Detection, Confidence-Based Anomaly Detection and Frequency-Based Outlier Detection with Back-off Smoothing in Textual Tabular Data}
\subtitleCZ   {Detekce špatné klasifikace, detekce anomálií na základě míry jistoty a detekce anomálií na základě četností v textových tabulkových datech}
            % \subtitle is optional
\author     {Bc. Jakub Ambroz}
\authorinfo {ambrojak@fel.cvut.cz}
\date       {September 2024, May 2025} 
\supervisor {Ing. Martin Komárek
%\nl Supervisor-specialist: Ing. Martin Komínek % Not sure if to add or not
}  % One or more supervisors
\studyinfo  {Open Informatics, specialization Data Science}  % Study programme etc.
% \workname   {Software or Research project} % Used only if \worktype [O/*] (Other)
            % optional more information about the document:
\workinfo   {\url{TODO GitHub Link}\rfc{Copy gitlab repo to public GitHub}}
            % Title / Subtitle in minor language:
\titleCZ    {Detekce anomálií v transportních objednávkách firmy Metrans}
\subtitleCZ {}
            % If minor language is other than English
            % use \titleCZ, \subtitleCZ or \titleSK, \subtitleSK instead it.
\pagetwo    {}  % The text printed on the page 2 at the bottom.

\abstractEN {
This thesis addresses the problem of detecting anomalies in transport orders processed by Metrans, a major logistics provider in Central Europe. By analyzing historical order data, we develop machine learning models capable of identifying irregularities such as incorrect heating types, missing customs identifiers, or atypical chassis assignments.

Three distinct detection approaches are proposed and evaluated: approach based on classification, out-of-distribution detection, and confidence-based anomaly scoring. The final solution is built as a modular microservice-oriented application deployed via CodeNOW and integrates tools like MLflow and Docker for reproducibility and scalability.
\rfc{Před odevzdáním ještě upravit abstract, aby byl v tipťop stavu}
   
}
\abstractCZ {
Tato diplomová práce se zabývá detekcí anomálií v přepravních objednávkách zpracovávaných společností Metrans, významným poskytovatelem logistických služeb ve střední Evropě. Na základě analýzy historických dat o objednávkách jsou vyvinuty modely strojového učení schopné identifikovat nepravidelnosti, jako jsou nesprávné typy ohřevu, chybějící celní identifikátory nebo neobvyklé typy podvozků.

Jsou navrženy a vyhodnoceny tři odlišné přístupy k detekci anomálií — klasifikační modely, detekce anomálií na základě četnosti a detekce na základě míry jistoty modelu. Výsledné řešení je implementováno jako modulární aplikace postavená na mikroservisní architektuře a nasazeno prostřednictvím platformy CodeNOW s využitím nástrojů jako MLflow a Docker pro zajištění reprodukovatelnosti a škálovatelnosti.
\rfc{Před odevzdáním ještě upravit abstrakt, aby byl v tipťop stavu}
   
}
   

\keywordsEN {%
Anomaly detection in tabular data, Misclassification Detection, Frequency-Based Outlier Detection, Confidence-Based Anomaly Detection
\rfc{Zkontrolovat keywords, doplnit kdyžtak něco}
}
\keywordsCZ {%
Detekce anomálií v tabulkových datech, Detekce špatné klasifikace, Detekce anomálií na základě četností, Detekce anomálií na základě míry jistoty
\rfc{Zkontrolovat, že překlad klíčových slov dává smysl}
}
\thanks {           % Use main language here
   I would like to thank Stratox, and especially Martin Komínek, for their invaluable support in resolving the technical challenges I encountered.
   
   I am also grateful to Metrans for providing real-world data essential to this project and to their employees for their guidance in contextualizing the data and linking it to operational meaning.
}
\declaration {      % Use main language here
    I declare that I have prepared the final thesis independently and have listed all information sources used in accordance with the Methodological Guidelines on the observance of ethical principles in the preparation of university final theses and the Framework Rules for the use of artificial intelligence at CTU for study and pedagogical purposes in Bc and NM studies.
    
    I declare that I have used artificial intelligence tools during the preparation and writing of the final thesis. I have verified the generated content. I confirm that I am aware that I am fully responsible for the content of the final thesis.

   \nl
   (CZECH VERSION)
   
   Prohlašuji, že jsem diplomovou práci vypracoval samostatně a uvedl veškeré použité informační zdroje v souladu s Metodickým pokynem o dodržování etických principů při přípravě vysokoškolských závěrečných prací a Rámocovými pravidly používání umělé inteligence na ČVUT pro studijní a pedagogické účely v Bc a NM studiu.
    
    Prohlašuji, že jsem v průběhu příprav a psaní závěrečné práce použil nástroje umělé inteligence. Vygenerovaný obsah jsem ověřil. Stvrzuji, že jsem si vědom, že za obsah závěrečné práce plně zodpovídám. 

    \nl\nl\nl
    
    
    Bc.~Jakub Ambroz
    
    ........................................

    In~Prague 23.05.2025
    
    (V~Praze dne 23.05.2025)

}

%%%%% <--   % The place for your own macros is here.

\draft     % Uncomment this if the version of your document is working only.
%\linespacing=1.7  % uncomment this if you need more spaces between lines
                   % Warning: this works only when \draft is activated!
%\savetoner        % Turns off the lightBlue backround of tables and
                   % verbatims, only for \draft version.
%\blackwhite       % Use this if you need really Black+White thesis.
%\onesideprinting  % Use this if you really don't use duplex printing. 
%\totf             % More compact list of tables and figures.

\input ambrojak-glosdata

\specification {
\vbox to0pt{\vskip-25mm\centerline{\inspic assignments/assignment-page1.pdf }\vss}
\vfil\break
\vbox to0pt{\vskip-25mm\centerline{\inspic assignments/assignment-page2.pdf }\vss}
\vfil\break
\vbox to0pt{\vskip-25mm\centerline{\inspic assignments/declaration.pdf }\vss}
}

\makefront  % Mandatory command. Makes title page, acknowledgment, contents etc.

\input tex/introduction
\input tex/anomalies
\input tex/methods
\input tex/approaches
\input tex/implementation
\input tex/results
\input tex/component
\input tex/final

\bibchap
\usebib/c (iso690) ambrojak-bib


\app Glossary
\makeglos

\input tex/appendices

\bye
