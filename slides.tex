\input ctustyle3

\let\s=\relax \let\restore=\relax

\worktype[O/EN]
\faculty{F3}
\department{Department of Mathematics}

\activettchar`

\slides

\slideshow 

\tit CTUslides 

\subtit simple slides in CTUstyle design\nl using \OpTeX/

\bigskip
\subtit\Black Petr Olšák\nl petr@olsak.net

\subtit\rm \url{http://petr.olsak.net/ctustyle.html}

\pg; %------------------------------------------------------------------

\sec Basics

* The document is included in a file (say `file.tex`)\nl
  and it can be processed by `optex file` command.
* The header of the document should be:

\begtt
\input ctustyle3  % CTUstyle macro for OpTeX
\slides           % slides initialization
\worktype[B/EN]   % type of the work (B,M,D,O) and language (CZ,SK,EN)
\faculty{F3}      % the faculty in short
\department {Department of Mathematics} % department

\slideshow        % begin of the document
... document ...
\pg.
\endtt

* The document must be finished by `\pg` followed by period.

* You need OpTeX in the version Jun~2020 or newer.\nl
  See \url{http://petr.olsak.net/optex}.

* The work type should be set similarly as in {\bf\Blue CTUstyle}.

* Only `\worktype`, `\faculty` and `\department`
  work here. No more declaration sequences from {\bf\Blue CTUstyle}.

\pg; %------------------------------------------------------------------

\sec The structural commands

* You can type \code{\*} for starting of the item.
* Nested items lists (second and more level) are created in
  the `\begitems`\dots`\enditems` environments.
* The slide titles are created by `\sec Text` 
  followed by end of line.\nl 
  For subsections, you can use `\secc Text` similarly.
* The title page (first slide) can be special if `\tit Title`\nl
  (followed by end of line) is used here.
* The `\subtit Author name etc.` (followed by end of line)
  can be used after `\tit` at the first slide.
* The paragraph texts are ragged right.
* You can use `\nl` for new line in the paragraph or titles.
* You can use `\pg` followed by \code{+} or \code{;} or \code{.}
  for new slide.
* The page-bar in the right corner is clickable and it will be 
  created correctly after second pass of the \TeX{} run.

\pg; %------------------------------------------------------------------

\sec Next page (next slide)

* The control sequence \code{\\pg} must be followed by:

\begitems
* the character \frame{\strut\code{+}} -- next page keeps the same text
  and a next text is added (usable for partially uncovering of ideas),\pg+ 
* the character \frame{\strut\code{;}} -- normal next page,\pg+
* the character \frame{\strut\code{.}} -- the end of the document.
\enditems
\pg+

* Summary:
\begtt
\pg+    ... uncover next text
\pg;    ... next page
\pg.    ... the end of the document
\endtt
\pg+

* If the control sequence `\slideshow` is removed (or commented out)
  from the beginning of the document then `\pg+` sequences are deactivated.
  This is usable for printing version of the document.

\pg; %------------------------------------------------------------------

\sec Verbatim

\secc Verbatim in paragraph

* Unlike {\Blue\bf CTUstyle} for OPmac, you can use 
  `"code text"` inside paragraph directly.
* If you declare \code{\\activettchar`} before `\slides` then you can use
  \code{`code text`} like in Markdown. 
* You can use `\code{text}` too.
* All these features are described in \OpTeX/ documentation. 

\pg+

\secc Multi-line verbatim

* Unlike {\Blue\bf CTUstyle} for OPmac, you can use the pair
  `\begtt...\endtt` directly as described in \OpTeX/ documentation.
  No `\pg=` is needed.
* Of course, you can use `\verbinut` too, if you want.

\pg; %------------------------------------------------------------------

\sec Example of multi-line verbatim

The source code includes:

\begtt \adef|{\_bslash}
|begtt |hisyntac{C}
#include <stdio.h>
int main(); // This is a program in C
{
  printf("Hello world!\n");
}
|endtt
\endtt

\pg+

and the result is:

\begtt \hisyntax{C}
#include <stdio.h>
int main(); // This is a program in C
{
  printf("Hello world!\n");
}
\endtt

Note that local declarations can be inserted after `\begtt`,\nl
the `\hisyntax` declaration is used in the example here.

\pg; %------------------------------------------------------------------

\sec Limits of the \code{\\pg+} sequence

* The \code{\\pg+} sequence cannot be used inside a group.
* The exception is the nested environment \code{\\begitems...\\enditems}.\pg+

\secc What to do?

* If you need to partially uncover the multi-line verbatim\nl
  then you can use:

\begtt \adef/{\bslash}
\begtt
... first line of the code ...
/endtt 
\par\pg+ \vskip-6.75pt
\begtt
... second line of the code ...
/endtt
\endtt

\pg+


* If you need to uncover the texts more ingenious then you can
  use `\layers...\endlayers` environment (see next slide\dots)

\pg; %------------------------------------------------------------------

\sec Uncovering by `\layers`, `\pshow`

* You can declare layers inside a slide by `\layers` $n$ ... `\endlayers` pair.
  The number $n$ declares the number of layers.

* The page with `\layers...\endlayers` pair is repeated $n$-times.

* You can use `\pshow` $k$ inside `\layers...\endlayers` environment.\nl
  This macro means {\em partially show}. It prints the following text to the end
  of current group:

\begitems
* invisible, if the number of the slide layer is less than $k$,
* red, if the number of slide the layer is equal to $k$,
* normal (black), if the number of slide layers is greater than $k$. 
\enditems

* The verbatim text and `\secc` macros canot be used inside 
  `\layer` environment.

* See \OpTeX/ documentation for more information.

* Next slide shows the usage of `\pshow`.

\pg; %------------------------------------------------------------------

\sec Example of `\pshow` usage

\setbox0=\vbox{
\begtt
\secc Ideas in special order

\layers 3
* {\pshow1 First idea}
* {\pshow3 Second idea}
* {\pshow2 Third idea}
\endlayers
\pg+

\secc A formula

\layers 4
Consider
$$ 
  E = {\pshow2 m}{\pshow3 c^2}
$$
\endlayers
That is great!
\pg;

\endtt
}
\puttext 100mm -120mm {\box0}

\bgroup
\hsize=8cm

\secc Ideas in special order

\layers 3
* {\pshow1 First idea}
* {\pshow3 Second idea}
* {\pshow2 Third idea}
\endlayers
\egroup
\pg+

\bgroup
\hsize=8cm 
\secc A formula
\egroup

\layers 4
\bgroup
Consider \hsize=8cm
$$ 
  E = {\pshow2 m}{\pshow3 c^2}
$$
\egroup
\endlayers
That is great!


\pg; %------------------------------------------------------------------

\restore

\sec Tables, pictures

* Tables can be created by `\table` or `\ctable` macro.
* Pictures can be included by \code{\\inspic} macro.
* See {\Blue\bf CTUstyle} and \OpTeX/ documentation for more details.
* The centering would be done by the `\centerline` macro.
* Example:\pg+

\begtt
\centerline{\picw=5cm \inspic cmelak1.jpg }
\endtt

\medskip
\centerline{\picw=5cm \inspic cmelak1.jpg } 

\pg+

* You can use `\puttext` or `\putpic` macro for arbitrary positioning of
  texts or images. 

\pg; %------------------------------------------------------------------

\sec Comparison CTUslides with Beamer\fnote{\url{http://www.ctan.org/pkg/beamer}}

The \LaTeX{} package Beamer gives much more features and many themes
are prepared for Beamer, {\bf\Red but}

\pg+
* the user of Beamer is forced to {\em program} his/her document using 
  dozens of \code{\\begin{foo}} and \code{\\end{foo}} and many another
  programming constructions,
* on the other hand, plain \TeX{} gives you a possibility to simply 
  {\em write} your document with minimal markup. The result is more compact.
\pg+
* You need to read 250 pages of doc for understanding Beamer,
* on the other hand, you need to read only ten 
  slides\fnote{this eleventh slide isn't counted}
  and you are ready to use {\bf\Blue CTUslides}.
\pg+
* A notice for programmers: to create another individual typographical 
  design for \LaTeX{} is much more complicated than to do the same
  in plain \TeX. And you need to seriously understand plain \TeX{} if you
  want to do something more complicated in \LaTeX.

\pg; %------------------------------------------------------------------

\null
\vskip2cm
\centerline{\typosize[35/40]\bf Thanks for your attention}\pg+

\vskip2cm
\centerline{\Blue\typosize[60/70]\bf Questions?}

\pg. %------------------------------KONEC-------------------------------

